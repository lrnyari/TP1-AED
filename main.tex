\documentclass[10pt,a4paper]{article}

\input{AEDmacros}
\usepackage{caratula} % Version modificada para usar las macros de algo1 de ~> https://github.com/bcardiff/dc-tex


    \titulo{Trabajo práctico 1: Especificación y WP}
\subtitulo{"En búsqueda del camino"}

\fecha{\today}

\materia{Algoritmos y Estructuras de Datos}
\grupo{Grupo NRO}

    \integrante{Roko, Tomás Esteban}{262/23}{tomas.e.roko@gmail.com}
\integrante{Nyari, Lisandro Rafael}{773/24}{lisandronyari@gmail.com}
\integrante{Apellido, Nombre3}{003/01}{email3@dominio.com}
\integrante{Apellido, Nombre4}{004/01}{email4@dominio.com}

\graphicspath{{../static/}}

% ACÁ TERMINA LA PRIMERA CARILLA 

\begin{document}

% ACÁ ARRANCA LA 2DA CARILLA

\maketitle

\section{Especificación}

\subsection{grandesCiudades}

A partir de una lista de ciudades, devuelve aquellas que tienen más de 50.000 habitantes.

\vspace{2mm}  % DEJO 2mm de espacio VERTICAL (por eso 'V'space). Si quieren dejar espacio en horizontal usan hspace{}

\begin{proc}{grandesCiudades}{\In ciudades : \TLista{Ciudad}}{\TLista{Ciudad}}
	\requiere{True}
	\asegura{todasMayoresA50k(res) = True} 
\end{proc}


\vspace{2mm}

\pred{todasMayoresA50k}{ciudades : \TLista{Ciudad}}{\paraTodo[unalinea]{i}{\ent}{(0 \leq i < |ciudades|) \implicaLuego ciudades[i][1] > 50000}}
% ¡¡IMPORTANTE!!! Es necesario indexar DOS VECES la lista 'ciudades' porque como dice el tp, está compuesta por TUPLAS < nombre, #habitantes>                                                                                                           

\vspace{4mm}

\subsection{sumaDeHabitantes}

Por cuestiones de planificación urbana, las ciudades registran sus habitantes mayores de edad por un lado y menores de edad por el otro. Dadas dos listas de ciudades del mismo largo con los mismos nombres, una con sus habitantes mayores y otra con sus habitantes menores, este procedimiento debe devolver una lista de ciudades con la cantidad total de sus habitantes.

\vspace{2mm}

\begin{proc}{sumaDeHabitantes}{\In menoresDeCiudades : \TLista{Ciudad}, \In mayoresDeCiudades: \TLista{Ciudad}}{\TLista{Ciudad}}
	\requiere{}
	\asegura{} 
\end{proc}


\vspace{3cm}


\subsection*{Macros de la cátedra para especificar(CON ESTAS DE BASE NOSOTROS NOS ARMAMOS NUESTRAS PRED, AUX, PROD)}

\begin{proc}{nombre}{\In paramIn : \nat, \Inout paramInout : \TLista{\ent}}{tipoRes}
	%    \modifica{parametro1, parametro2,..}
	\requiere{expresionBooleana1}
	\asegura{expresionBooleana2}
	\aux{auxiliar1}{parametros}{tipoRes}{expresion}
	\pred{pred1}{parametros}{expresion} 
\end{proc}

\aux{auxiliarSuelto}{parametros}{tipoRes}{expresion}
% \paraTodo{variable}{tipo}{expresion}
% \existe{variable}{tipo}{expresion}
% Pueden tener [unalinea] para que no se divida en varias lineas
\pred{predSuelto}{parametros}{\paraTodo[unalinea]{variable}{tipo}{algo \implicaLuego expresion}}
\pred{predSuelto}{parametros}{\existe[unalinea]{variable}{tipo}{algo \yLuego expresion}}



\vspace{10cm}


\end{document}
